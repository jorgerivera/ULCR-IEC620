\documentclass[handout,xcolor=dvipsnames]{beamer}

\usepackage[utf8x]{inputenc}
\usepackage{default}
\usetheme[width=50pt]{ULatina}   % Bergen, Darmstadt
\usecolortheme[named=Green]{structure}
\usepackage{graphicx}
\usepackage{pgfpages}
\usepackage{tikz}
\usepackage[spanish]{babel}

\newcommand{\uC}{\ensuremath{\mu\textrm{C}}\ }
\newcommand{\uP}{\ensuremath{\mu\textrm{P}}\ }
%\setbeameroption{handout}
%\setbeameroption{show notes on second screen}

\title[\uP]{Microprocesadores}
\subtitle{Caracter\'\i sticas B\'asicas}
\author{Prof. Jorge Rivera~Guti\'errez}
\institute{Universidad Latina de Costa Rica\\ Ingenier\'\i a en Electr\'onica}
\logo{\includegraphics[height=45pt]{world.png}}
\newcommand{\pageframe}[1]{\frame{\begin{center}{ \Huge #1 }\end{center}}}


\begin{document}

\begin{frame}
 \maketitle
\end{frame}

\begin{frame}
 \tableofcontents
\end{frame}

\section{Repaso de Arquitectura}

\begin{frame}[t]
  \frametitle{Definición}
  \begin{center}
    ~\vfill~\\
    \Huge ¿Qué es arquitectura de computadores?\\
    ~\vfill~\\
  \end{center}

\end{frame}
\subsection{Arquitectura}
\begin{frame}[t]
  \frametitle{Definición}

  \only<1->{
  \begin{block}{Arquitectura de Instrucciones, ISA}
    La porción de la computadora que es visible al progamador o quien escribe un compilador.
    \note<1>[item]{Clase: registros de propósito general (registro-memoria o load-store).}
    \note<1>[item]{Direccionamiento de memoria: por bytes. Alineamiento.}
    \note<1>[item]{Modos de direcc: registro, inmediato, desplazamiento}
    \note<1>[item]{Tipos y tamaños de operandos}
    \note<1>[item]{Operaciones: transf, aritméticas, lógicas, control, floatp...}
    \note<1>[item]{Instrucciones de Control de flujo: cond/uncond branch, calls, return}
    \note<1>[item]{Codificación: fixed-len vs variable}
  \end{block}}
  \only<2->{\begin{block}{Organización}
    La forma en que los componentes de hardware se conectan para formar un sistema computacional. 
    \note<2>[item]{Conexión externa de la memoria}
    \note<2>[item]{Forma en que se ejecuta cada instrucción internamente}
    \note<2>[item]{AMD Opteron 64 vs Intel Pentium 4}
    \note<2>[item]{Pipeline, cache, etc.}
  \end{block}}
  \only<3->{\begin{block}{Diseño de Hardware}
    Las partes específicas de la computadora: diseño lógico detallado y empaquetamiento de la tecnología de la computadora.
    \note<3>[item]{Diseño y selección de componentes para una tarjeta madre}
    \note<3>[item]{Ej: versión desktop vs móvil}
  \end{block}}

\end{frame}
\subsection{Modelos de diseño}
\pageframe{Modelos de diseño}

\pageframe{Harvard}

\begin{frame}[t]
  \frametitle{Harvard}
  \begin{block}{Descripción}
   Utiliza memorias separadas para almacenar datos e instrucciones, cada una conectada a un bus distinto.
  \end{block}
  \only<2->{
  \begin{exampleblock}{Esquema}
   \centering\input{diagramas/1-harvard.tex}
  \end{exampleblock}}
 
\end{frame}

\pageframe{von Neumann}

\begin{frame}[t]
  \frametitle{von Neumann}
  \begin{block}{Descripción}
   Utiliza una sola memoria física para almacenar datos e instrucciones.
  \end{block}
  \only<2->{
  \begin{exampleblock}{Esquema}
   \centering\input{diagramas/1-vonNeumann.tex}
  \end{exampleblock}}
\end{frame}


\begin{frame}{Otros conceptos}
  \begin{block}{Conceptos}
   \begin{itemize}
    \item<1-> Segmentada: se subdivide la memoria en segmentos, dentro de los cuales residen código, datos, pila, extras. También se refiere al uso de memoria ``protegida''.
      \note<1->[item]{Divide la memoria en segmentos para ampliar la capacidad de memoria. 8086 16/16+4bits para ampliar la capacidad de memoria a 1MByte, pero se apuntan con registros de 16bits. El 286 (16/24) y 386 (32/32) amplían la memoria, pero conservan los modos de direccionamiento básicos.} 
    \item<2-> Multi-etapa (pipeline). Ejecución de múltiples instrucciones simultáneamente al dividir las instrucciones en etapas.
      \note<2->[item]{Se dividen las instrucciones en etapas (p. ej. instruction fetch, instruction decode and register fetch, execute, memory access, register write back) y se ejecutan instrucciones simultáneamente cuando es posible.}
   \end{itemize}

  \end{block}
\end{frame}



\subsection{Arquitectura de instrucciones}
\begin{frame}{Definición}
 \begin{block}{Arquitectura de instrucciones}
  ISA (\textit{Instruction Set Architecture}) es la parte de la arquitectura relacionada a la programación, y que define los tipos de datos, las instrucciones, los registros, los modos de direccionamiento, el manejo de interrupciones, excepciones, entradas y salidas.
 \end{block}

\end{frame}
\begin{frame}{CISC}
  \begin{block}{Complex Instruction Set Computer}
    \begin{itemize}
      \item<1-> Cada instrucción puede ejecutar una o más operaciones de bajo nivel, como carga de memoria a registro, operaciones aritméticas, almacenamiento en memoria.
	\note<1->[item]{Ejemplos: x86, 68k, PDP-11.}
      \item<2-> Intenta cerrar la brecha semántica.
	\note<2->[item]{Acercar el ensamblador a los lenguajes de alto nivel, haciéndolo más ``fácil'' para los programadores.}
      \item<3-> Disminuye el tamaño de los programas.
	\note<3->[item]{Muy importante en el tiempo del x86 original y otros procesadores CISC, donde la capacidad de almacenamiento y de memoria principal era muy reducida.}
    \end{itemize} 

  \end{block}

 
\end{frame}

\begin{frame}{RISC}
  \begin{block}{Reduced Instruction Set Computer}
   \begin{itemize}
    \item<1-> Las instrucciones son reducidas, realizan un número limitado de operaciones de bajo nivel.
      \note<1->[item]{No se refiere a un conjunto reducido, sino a instrucciones reducidas.}
    \item<2-> Modos de direccionamiento sencillo.
    \item<3-> Registros de propósito general.
    \item<4-> Pocos tipos de datos en hardware.
      \note<4->[item]{Algunos CISC tienen manejo de byte string o de números complejos, RISC no.}
      \note<4->[item]{ARM (GBA, DSi, iPod, etc), PPC, MIPS, SPARC, SuperH, AVR}
   \end{itemize}
  \end{block} 
\end{frame}

\section[$\mu C$]{Microcontroladores}

\begin{frame}
   \frametitle{Microcontroladores}
  \only<1>{\begin{center}
	      ~\vfill~\\
	      \Huge ¿Qué son los microcontroladores?\\
	      ~\vfill~\\
           \end{center}}

\end{frame}


\subsection{Descripción}
\pageframe{Descripción}
\begin{frame}[t]
  \frametitle{Características}
  \begin{block}{}
    \begin{itemize}[<+->]
      \item Recursos de entrada/salida \note<+->[item]{I/O, periféricos específicos, ADC, etc}
      \item Espacio optimizado \note<+->[item]{número de terminales, encapsulado}
      \item \uC idóneos: familias grandes con miles de opciones para escoger la que mejor se amolda
      \item Seguridad en el funcionamiento \note<+->[item]{WDT, etc}
      \item Bajo consumo \note<+->[item]{fuente de alimentación: batería}
      \item Protección de los programas \note<+->[item]{read protect}
    \end{itemize}

  \end{block}
 
\end{frame}
\begin{frame}[t]
  \frametitle{Ventajas}
  \only<1>{\begin{center}
	      ~\vfill~\\
	      \Huge ¿Qué ventajas ofrecen los microcontroladores?\\
	      \note{Preguntar a los estudiantes}
	      ~\vfill~\\
           \end{center}}

\end{frame}

\begin{frame}{¿Dónde encontramos microcontroladores?}
  \begin{block}{Consumo}
    \begin{itemize}
    \item Televisión, DVD, VCR, Cable, DVR
    \item PDAs, Celulares, Cámaras, GPS, Reproductores
    \item Electrodomésticos inteligentes
    \item Juguetes
    \end{itemize}
  \end{block}

  \begin{columns}[t]
    \column{.45\textwidth}
      \begin{block}{Automóviles}
       \begin{itemize}
        \item Inyección electrónica
	\item Frenos
	\item Control de vidrios
	\item Control de asientos
       \end{itemize}
      \end{block}
    \column{.45\textwidth}
      \begin{block}{Otros campos}
       \begin{itemize}
        \item Medicina
	\item Control industrial
	\item Redes
	\item Equipo de oficina
       \end{itemize}
      \end{block}
  \end{columns}

  \note{Negocio que mueve \$12.3b}
 
\end{frame}


\subsection{Periféricos}
\pageframe{Periféricos}
\pageframe{Periféricos:\\~\\GPIO}
\begin{frame}[t]
  \frametitle{Puertos de I/O}

  \begin{block}{GPIO}
    \begin{itemize}[<+->]
      \item GPIO (\textit{General purpose input/output})
      \item Son pines que no tienen un propósito específico
	\note<+->[item]{Opuesto a los específicos como acceso a periféricos (ADC, PWM, comm)}
      \item Niveles de voltaje variables, acordes al \uC
      \item Dirección configurable
      \item Presentes en \uC y otros chips de propósito específico
    \end{itemize}
  \end{block}
\end{frame}

\pageframe{Periféricos:\\~\\ADC}
\begin{frame}[t]
  \frametitle{Convertidor Analógico a Digital}
  \begin{block}{ADC}
    \begin{itemize}[<+->]
      \item ADC (\textit{Analog-to-Digital Converter})
      \item Convierte un valor continuo en uno discreto.
    \end{itemize}
  \end{block}
  \begin{block}{Consideraciones}
    \begin{itemize}[<+->]
      \item Rango de valores digitales.
      \item Velocidad de muestreo.
      \item Número de canales.
      \item Canales independientes o multiplexados.
      \item Referencias: internas, externas, configurables.
    \end{itemize}

  \end{block}

\end{frame}


\pageframe{Periféricos:\\~\\PWM}
\begin{frame}[t]
  \frametitle{Modulador de ancho de pulso}
  \begin{block}{PWM}
    \begin{itemize}[<+->]
      \item PWM (\textit{Pulse-width Modulation})
      \item Genera una señal rectangular con un ancho de pulso proporcional a una variable provista.
    \end{itemize}


  \end{block}
  \begin{block}{Consideraciones}
    \begin{itemize}[<+->]
      \item Rango de valores digitales.
      \item Frecuencia de la señal. 
      \item Número de canales.
      \item Potencia de la señal de salida.
    \end{itemize}

  \end{block}
 
\end{frame}

\pageframe{Periféricos:\\~\\Comunicación}

\begin{frame}[t]
  \frametitle{Puertos de comunicación}
  \begin{block}{Comunicación}
    \begin{itemize}[<+->]
      \item Serial (RS-232, RS-485, )
	\note<.>[item]{}
      \item Paralelo
      \item I$^2$C, I$^2$S, SPI
      \item USB
    \end{itemize}

  \end{block}
  \begin{block}{Consideraciones}
    \begin{itemize}[<+->]
      \item Niveles de tensión.
	\note<.->[item]{Representación de 1s y 0s, diferencial vs. single-ended}
      \item Simetría, dirección y canales
	\note<.->[item]{Maestro, multi-maestro, esclavo.}
	\note<.->[item]{simplex, half-duplex, full-duplex}
      \item Sincronía
	\note<.->[item]{Síncrono, asíncrono}
      \item Buffers
	\note<.->[item]{tamaño, número, forma de uso}
    \end{itemize}

  \end{block}
 
\end{frame}

\end{document}
